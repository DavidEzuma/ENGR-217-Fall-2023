\documentclass[a4paper,12pt]{article}

\usepackage[margin=1in]{geometry}
\usepackage{tikz}
\usepackage{amssymb}
\usepackage{xcolor}
\usepackage{circuitikz}
\usepackage{graphicx}

\newcommand{\ra}{$\rightarrow$}
\newenvironment{6mini}{
  \begin{minipage}{6cm}
}{
  \end{minipage}
}

\title{\texttt{Electric Potential}\\\hrulefill}
\author{ENGR 217}
\date{\small{9/14/2023}}

\begin{document}
    \maketitle

    \section{Electric Potential Energy} 
        Coulomb force is conservative (dependent only on initial and final state)
            \begin{itemize}
                \item Independent of path
                \item Anaogous to gravity
            \end{itemize}
            Work done by a conservative force can be described as the difference in potential energy
            \[-\Delta U_{elec}=\Delta K\]
            \[W{ab}=U_a-U_b=-(U_b-U_a)=-\Delta U=q_0Ed\]
            Detemrine what force releative to the electric field, and the direciton of the displacement as well.
            \begin{itemize}
                \item doing work against the field (-W) is increases potential energy
            \end{itemize}
            
        \subsection*{Electric Potential Energy in a Uniform field}
            If the \textbf{positive} charge \underbar{moves in the direction of the field}, the field does positive work and the potential energy decreases. Thus, it's potential energy would \texttt{decrease} as it moves through the field. Moving it against the field with a force would \texttt{increase} it's potential energy.
                \begin{itemize}
                    \item reverse this concept for a negative charge
                \end{itemize}
            \[W_{ab}=Fd=qE_0\]

        \subsection*{Electric Potential energy of two poitn charges}
        \[U=k\frac{qq_0}{r}\]
        \begin{itemize}
            \item Units are joules
            \item It is a scalar qauntity
            \item Decreases as distance between charges approaches infinity
        \end{itemize}    
        
        \subsection*{Electric Potential energy of Multiple charges}
            \[U=q_0*k*\sum_{i}^{}\frac{q_i}{r}\]
            \subsubsection*{Total potential ENergy of a system of charges}
                \[U=k*\sum_{i<j}^{}\frac{q_iq_j}{r_{ij}}\]
        
        \section{ELectric Potential}    
        An electric potential (also called the electric field potential, potential drop, or the electrostatic potential) is the amount of work needed to move a unit of charge.
            \subsection*{Potential}
                \[\frac{W_{ab}}{q_0}=-\frac{\Delta U}{q_0}=-(\frac{U_b}{q_0}-\frac{U_a}{q_0})=-(V_b-V_a)=V_a-V_b\]
            \subsection*{Potential and the Electric field}
            \begin{itemize}
                \item if you move in the direction of the elctric field, the electric potential decreases
                \item In the opposite direction of the E field, the electric potential increases 
            \end{itemize}
            \subsection*{Due to a single charge}    
                \[V=\]
            \subsection*{converting potential to electric field}
                \[\vec{E}=-\]
                
\end{document}