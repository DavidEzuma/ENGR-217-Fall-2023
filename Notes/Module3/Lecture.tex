\documentclass[a4paper,12pt]{article}

\usepackage[margin=1in]{geometry}
\usepackage{tikz}
\usepackage{amssymb}
\usepackage{xcolor}
\usepackage{circuitikz}
\usepackage{graphicx}

\newcommand{\ra}{$\rightarrow$}
\newenvironment{6mini}{
  \begin{minipage}{6cm}
}{
  \end{minipage}
}

\title{\texttt{Coulomb's law}\\\hrulefill}
\author{ENGR 217}
\date{\small{9/7/2023}}

\begin{document}
    \maketitle

    \section{Charge}
        Property of matter that causes matter to attract or repel other matter. Additionally, Charge is quantized; the smallest/unit charge is that of an electron or proton. ie. $1.602 \cdot 10^{-19}$

        \subsection*{Coulumb's law}
            $\vec{F}=k\frac{q_1q_2}{r^2}\hat{r}$: where $\hat{r}$ is a unit vector pointing from q to $q_0$. k = 9.988$\times 10^9Nm^2/C^2$.

        \subsection*{Superposition}
        \[\vec{F}_{tot}=\sum_{i}^{}F_{q_i,q}\]
            *Coulomb's Law describes the interaction for two point charges. THis principle states that the total force is the \underbar{vector sum of all contributions of charges exerted individually}.
            \begin{itemize}
                \item Total electric field P is the vector sum of the fields at P due to each point charge in the distribution.                
            \end{itemize}

        \subsection*{Electric Field}
        A field is a physical quantity is assigned to every point in space.
        \begin{itemize}
            \item Tend to to extend over a volume and affect objects in said volume
            \item Quantity can be uniform or not throughout the entire volume
            \item Fields can have scalar or vector quantities
        \end{itemize}
        \[\vec{E}=\frac{\vec{F}}{q}\]
        Find the total force on the refernce charge, and that would be the forced used to the determine the electric field the charge is experiencing.
        \begin{itemize}
            \item Electric field will point towards negative charges and away from positve charges
            \item \(\vec{F}_{tot}=\sum_{i}^{}F_{q_i,q}\)
        \end{itemize}

        \subsection*{Work}
        General definition is “force acting through a distance.” (Always a scalar
        quantity). W=qE * d (work on a charge = Force on charge times distance).
        \textbf{ELectric Potential} is the amount of work needed to move a unit
        of charge.
        \begin{itemize}
            \item Written as V (or sometimes $\varphi$), measured in Volts
            \item work per unit charge
        \end{itemize}
        \includegraphics*[width=10cm]{Potential calc.png}
\end{document}